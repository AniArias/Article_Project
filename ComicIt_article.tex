\documentclass[12pt]{report}
\usepackage[spanish, activeacute]{babel}
\usepackage[top=2.75cm,bottom=2.50cm,left=3.00cm,right=2.50cm]{geometry}
\usepackage[utf8]{inputenc}  

\begin{document}
	\setlength{\topmargin}{-0.5in}
	\pagestyle{empty}
	\begin{center}
		\textbf{
			\vspace{-0.7em}
			ESCUELA SUPERIOR POLITÉCNICA DEL LITORAL
		}
		\line(1,0){380}\\		
		\scriptsize{FACULTAD DE INGENIERIA EN ELECTRICIDAD Y COMPUTACIÓN}
	\end{center}
	\begin{center}
		\vspace{2.5em}
		Lenguajes de Programación
		\\2012 | II Término
		\vspace{1.5em}
		\\Ana Arias - acarias@espol.edu.ec
		\\Liliana Ramos - ljramos@espol.edu.ec
		\\Denny Schuldt - dschuldt@espol.edu.ec
		\vspace{3em}
		\Large{\textbf{\\Comic It!	\vspace{2em}}}
	\end{center}
	\begingroup
		\large{
			\textbf{
				Descripción
				\newline
				\newline
			}
		}
	\endgroup
	%
	%REEMPLAZAR AQUÍ LA DESCRIPCIÓN-------------------------------------------------------------------------------------------------
	%
	Se llama historieta o cómic a una serie de dibujos que constituyen un relato, con texto o sin él,1 así como al medio de comunicación en su conjunto.2 Partiendo de la concepción de Will Eisner de esta narrativa gráfica como un arte secuencial, Scott McCloud llega a la siguiente definición: Ilustraciones yuxtapuestas y otras imágenes en secuencia deliberada con el propósito de transmitir información u obtener una respuesta estética del lector.3 Sin embargo, no todos los teóricos están de acuerdo con esta definición, la más popular en la actualidad, dado que permite la inclusión de la fotonovela4 y, en cambio, ignora el denominado humor gráfico.5
El interés por el cómic puede tener muy variadas motivaciones, desde el interés estético al sociológico, de la nostalgia al oportunismo.6 Durante buena parte de su historia fue considerado incluso un subproducto cultural,7 apenas digno de otro análisis que no fuera el sociológico, hasta que en los años 60 del pasado siglo se asiste a su reivindicación artística, de tal forma que Morris8 y luego Francis Lacassin9 han propuesto considerarlo como el noveno arte, aunque en realidad sea anterior a aquellas disciplinas a las que habitualmente se les atribuyen las condiciones de octavo (fotografía, de 1825) y séptimo (cine, de 1886). Seguramente, sean este último medio y la literatura los que más la hayan influido, pero no hay que olvidar tampoco que su particular estética ha salido de las viñetas para alcanzar a la publicidad, el diseño, la moda y, no digamos, el cine.
Las historietas suelen realizarse sobre papel, o en forma digital (e-comic, webcómics y similares), pudiendo constituir una simple tira en la prensa, una página completa, una revista o un libro (álbum, novela gráfica o tankobon). Han sido cultivadas en casi todos los países y abordan multitud de géneros. Al profesional o aficionado que las guioniza, dibuja, rotula o colorea se le conoce como historietista.11
Se llama historieta o cómic a una serie de dibujos que constituyen un relato, con texto o sin él,1 así como al medio de comunicación en su conjunto.2 Partiendo de la concepción de Will Eisner de esta narrativa gráfica como un arte secuencial, Scott McCloud llega a la siguiente definición: Ilustraciones yuxtapuestas y otras imágenes en secuencia deliberada con el propósito de transmitir información u obtener una respuesta estética del lector.3 Sin embargo, no todos los teóricos están de acuerdo con esta definición, la más popular en la actualidad, dado que permite la inclusión de la fotonovela4 y, en cambio, ignora el denominado humor gráfico.5
	\newline
	\newline
	\newline
	\newline
	\begingroup
		\large{
			\textbf{
				Funcionalidades
				\newline
				\newline
			}
		}
	\endgroup
	%
	%REEMPLAZAR AQUÍ LAS FUNCIONALIDADES---------------------------------------------------------------------------------------------
	%
Se llama historieta o cómic a una serie de dibujos que constituyen un relato, con texto o sin él,1 así como al medio de comunicación en su conjunto.2 Partiendo de la concepción de Will Eisner de esta narrativa gráfica como un arte secuencial, Scott McCloud llega a la siguiente definición: Ilustraciones yuxtapuestas y otras imágenes en secuencia deliberada con el propósito de transmitir información u obtener una respuesta estética del lector.3 Sin embargo, no todos los teóricos están de acuerdo con esta definición, la más popular en la actualidad, dado que permite la inclusión de la fotonovela4 y, en cambio, ignora el denominado humor gráfico.5
	\newline
	\newline
	\newline
	\newline
	\begingroup
		\large{
			\textbf{
				Demo	
				\newline
				\newline
			}
		}
	\endgroup
\end{document}
