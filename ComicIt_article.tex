\documentclass[12pt]{report}
\usepackage[spanish, activeacute]{babel}
\usepackage[top=2.75cm,bottom=2.50cm,left=3.00cm,right=2.50cm]{geometry}
\usepackage[utf8]{inputenc}  
\usepackage{enumerate}


\begin{document}
	\setlength{\topmargin}{-0.5in}
	\pagestyle{empty}
	\begin{center}
		\textbf{
			\vspace{-0.7em}
			ESCUELA SUPERIOR POLITÉCNICA DEL LITORAL
		}
		\line(1,0){380}\\		
		\scriptsize{FACULTAD DE INGENIERÍA EN ELECTRICIDAD Y COMPUTACIÓN}
	\end{center}
	\begin{center}
		\vspace{2.5em}
		Lenguajes de Programación
		\\2012 | II Término
		\vspace{1.5em}
		\\Ana Arias - acarias@espol.edu.ec
		\vspace{2mm}
		\\Liliana Ramos - ljramos@espol.edu.ec
		\\Denny Schuldt - dschuldt@espol.edu.ec
		\vspace{3em}
		\Large{\textbf{\\Comic It!	\vspace{2em}}}
	\end{center}
	


	%
	%Objetivos
	%
	\begingroup
		\large{
			\textbf{
				Objetivo General
				\newline
				\newline
			}
		}
	\endgroup

	Definir y dar a conocer las funcionalidades y los requerimientos que tendrá el proyecto para la materia Lenguajes de Programación de la Escuela Superior Politécnica del Litoral. 
Constrastar y relatar las experiencias vividas con GitHub y LaTeX.
	\vspace{7mm}

	\begingroup
		\large{
			\textbf{
				Objetivos Específicos
				\newline
				\newline
			}
		}
	\endgroup
		\begin{enumerate}[(a)]%for small alpha-characters within brackets.
		\item Conocer las funcionalidades de la nueva aplicación para Android: Comic It!.
		\item Describir cada una de las funcionalidades que tendrá la aplicación.
		\item Especificar quiénes serán los usuarios finales de la aplicación.
		\item Presentar ciertas características que tendrá la aplicación final.
		\item Exponer experiencias con la instalación/utilización de GitHub y LaTex.
		\item Relatar anécdotas con la instalación/utilización de GitHub y LaTex.
		\end{enumerate}
	
	\vspace{7mm}

\newpage

	\begingroup
		\large{
			\textbf{
				Descripción
				\newline
				\newline
			}
		}
	\endgroup
	%
	%Descripcion
	%
``Comic It!'' es una nueva aplicación para disposivos móviles que utilizan como sistema operativo Android. El nombre representa completamente a este nuevo software, puesto que será diseñado para crear historietas. ``Comic It!'' tendrá como usuarios a personas que les gusta tomar fotos y conservar recuerdos de los momentos divertidos que viven diariamente.
\newline
\newline
Con plantillas para colocar sus fotos, burbujas de diálogo e imágenes predeterminadas, el usuario quedará satisfecho cuando obtenga su obra final en su dispositivo. Esta historieta relatará de una manera divertida, humorística y muy colorida exactamente lo que el usuario ha vivido o haya querido inventar para su uso personal.
	\newline
	\newline
	\newline
	\newline




\begingroup
		\large{
			\textbf{
				Funcionalidades
				\newline
				\newline
			}
		}
	\endgroup
	%
	%REEMPLAZAR AQUÍ LAS FUNCIONALIDADES---------------------------------------------------------------------------------------------
	%
Se llama historieta o cómic a una serie de dibujos que constituyen un relato, con texto o sin él,1 así como al medio de comunicación en su conjunto.2 Partiendo de la concepción de Will Eisner de esta narrativa gráfica como un arte secuencial, Scott McCloud llega a la siguiente definición: Ilustraciones yuxtapuestas y otras imágenes en secuencia deliberada con el propósito de transmitir información u obtener una respuesta estética del lector.3 Sin embargo, no todos los teóricos están de acuerdo con esta definición, la más popular en la actualidad, dado que permite la inclusión de la fotonovela4 y, en cambio, ignora el denominado humor gráfico.5
\newline
	\newline
	\newline
	\newline

	\begingroup
		\large{
			\textbf{
				Demo	
				\newline
				\newline
			}
		}
	\endgroup



	\begingroup
		\large{
			\textbf{
				Experiencias y Anécdotas: GitHub
				\newline
				\newline
			}
		}
	\endgroup


	\begingroup
		\large{
			\textbf{
				Experiencias y Anécdotas: LaTeX
				\newline
				\newline
			}
		}
	\endgroup
\end{document}


