\documentclass[12pt]{report}
\usepackage[spanish, activeacute]{babel}
\usepackage[top=2.75cm,bottom=2.50cm,left=3.00cm,right=2.50cm]{geometry}
\usepackage[utf8]{inputenc}  
\usepackage{enumerate}
\usepackage{graphicx}


\begin{document}
	\setlength{\topmargin}{-0.5in}
	\pagestyle{empty}
	\begin{center}
		\textbf{
			\vspace{-0.7em}
			ESCUELA SUPERIOR POLITÉCNICA DEL LITORAL
		}
		\line(1,0){380}\\		
		\scriptsize{FACULTAD DE INGENIERÍA EN ELECTRICIDAD Y COMPUTACIÓN}
	\end{center}
	\begin{center}
		\vspace{2.5em}
		Lenguajes de Programación
		\\2012 | II Término
		\vspace{1.5em}
		\\Ana Arias - acarias@espol.edu.ec
		\vspace{2mm}
		\\Liliana Ramos - ljramos@espol.edu.ec
		\\Denny Schuldt - dschuldt@espol.edu.ec
		\vspace{3em}
		\Large{\textbf{\\Comic It!	\vspace{2em}}}
	\end{center}	
	\begingroup
		\large{
			\textbf{
				Objetivo General
				\newline
				\newline
			}
		}
	\endgroup
	Definir y dar a conocer las funcionalidades y los requerimientos que tendrá el proyecto para la materia Lenguajes de Programación de la Escuela Superior Politécnica del Litoral. 
Constrastar y relatar las experiencias vividas con GitHub y LaTeX.
	\vspace{4em}
	\newline
	\begingroup
		\large{
			\textbf{
				Objetivos Específicos
				\newline
			}
		}
	\endgroup
		\begin{enumerate}[(a)]%for small alpha-characters within brackets.
		\item Conocer las funcionalidades de la nueva aplicación para Android: Comic It!.
		\item Describir cada una de las funcionalidades que tendrá la aplicación.
		\item Especificar quiénes serán los usuarios finales de la aplicación.
		\item Presentar ciertas características que tendrá la aplicación final.
		\item Exponer experiencias con la instalación/utilización de GitHub y LaTex.
		\item Relatar anécdotas con la instalación/utilización de GitHub y LaTex.
		\end{enumerate}
	
	\vspace{7mm}
	\begingroup
		\large{
			\textbf{
				Descripción
				\newline
				\newline
			}
		}
	\endgroup
	%
	%Descripcion
	%
``Comic It!'' es una nueva aplicación para disposivos móviles que utilizan como sistema operativo Android. El nombre representa completamente a este nuevo software, puesto que será diseñado para crear historietas. ``Comic It!'' tendrá como usuarios a personas que les gusta tomar fotos y conservar recuerdos de los momentos divertidos que viven diariamente.
\newline
\newline
Con plantillas para colocar sus fotos, burbujas de diálogo e imágenes predeterminadas, el usuario quedará satisfecho cuando obtenga su obra final en su dispositivo. Esta historieta relatará de una manera divertida, humorística y muy colorida exactamente lo que el usuario ha vivido o haya querido inventar para su uso personal.
	\newline
	\newline
	\newline
	\begingroup
		\large{
			\textbf{
				Funcionalidades
				\newline
				\newline
			}
		}
	\endgroup
	%
\newline
Comic It! cuenta con una variedad de funcionalidades que el usuario puede usar de manera rápida y dinámica utilizando 				fotos tomadas directamente de la CÁMARA o de la GALERÍA FOTOGRÁFICA .
El usuario tendrá la opción de crear un collage tipo caricatura  con fotografías de la galería de fotos del celular en una plantilla escogida de una LISTA DE PLANTILLAS proporcionadas por la aplicación.
Comic It! cuenta con una LISTA DE ÍCONOS básicos y personalizados que permitirán al usuario crear imágenes más reales y divertidas.		
Además de íconos, el usuario tendrá una LISTA DE TEXTOS divertidos para darle mas creatividad a las escenas que esté creando.
Proporciona una lista con varios formatos de BURBUJAS DE DIALOGO, de las cuales el usuario puede escoger las que más se ajuste a su necesidad. El usuario podrá EDITAR EL TEXTO dentro de la burbuja de dialogo que haya escogido, convirtiendo a los miembros de 		la fotografía en auténticos personajes.
Al terminar de crear caricaturas, se podrán guardar en la galería fotográfica en distintos FORMATOS.
	\newline
	\newline
	\newline
	\begingroup
		\large{
			\textbf{
				Ilustración	
				\newline
				\newline
			}
		}
	\endgroup
Este es un pequeño vistazo a como se espera que sea la aplicación, resaltando sus partes más importantes:
	\newline
	\newline
	\begin{center}
		\begingroup
			\includegraphics[width=0.19\textwidth]{demo/demo1.jpg}
		\endgroup
		\begingroup
			\includegraphics[width=0.19\textwidth]{demo/demo2.jpg}
		\endgroup
		\begingroup
			\includegraphics[width=0.19\textwidth]{demo/demo3.jpg}
		\endgroup
		\begingroup
			\includegraphics[width=0.19\textwidth]{demo/demo4.jpg}
		\endgroup
		\begingroup
			\includegraphics[width=0.19\textwidth]{demo/demo5.jpg}
		\endgroup
	\end{center}

Se observa la imagen inicial de la aplicación, seguido de las pantallas de selección de plantillas. Luego, puede verse un bosquejo de la aplicación en su parte más importante: La edición de la historieta.
 \newline
 \newline
En la parte superior, se tienen los botones 'Guardar' y 'Compartir'. Guardar, como su nombre lo indica, permitiría guardar la aplicación en la librería de imágenes del smartphone. 'Compartir' permitiría mostrar la historieta finalizada en las redes sociales, Twitter y Facebook.
 \newline
 \newline
En la parte inferior se observan los botones '-', '+'  y el botón de selección de burbujas. Los dos primeros botones servirían para redimensionar las burbujas o íconos en pantalla.
           \newline
           \newline
	\begingroup
		\large{
			\textbf{
			           \newline
			           \newline
				Experiencias y Anécdotas: GitHub
				\newline
				\newline
			}
		}
	\endgroup
	\textbf{Ana:\newline\newline}Mi experiencia con GitHub ha sido buena, esta herramienta en verdad mejoró la comunicación en los trabajos con mi grupo; lo que más me agradó es que podemos estar actualizados con respecto a las modificaciones en los archivos que nos tocó hacer de forma grupal, y por lo tanto se evita el compartir documentos frecuentemente a través de otra herramienta y la creación de un sinfín de documentos que contienen en realidad lo mismo.
\newline
\newline	
Considero Github como una red social, ya que podemos estar conectados con nuestros compañeros y compartir información, y además de ser social, es seria ya que a partir de esta herramienta se pueden encontrar trabajos muy interesantes de distintas personas, dicho material puede ser muy útil.
\newline
\newline	
Con respecto a su uso, lo considero sencillo y mecánico; al principio es un poco difícil aprender cómo se utiliza pero después se vuelve costumbre ya que los procedimientos para subir archivos, para crear repositorios, para modificarlos es en realidad muy mecánico.
\newline
Esta herramienta es una buena opción para dar a conocer nuestros trabajos en internet, y seguramente la seguiré utilizando en el futuro.
\newline
\newline
\textbf{Denny:\newline\newline} Utilizar Github es una buena oportunidad para mantener organizados los proyectos que desarrollemos, y de igual manera, poder compartir nuestro código con más usuarios. 
\newline
\newline
Uno de los principales inconvenientes a la hora de utilizar Github, es el poco conocimineto que se puede tener en un principio, de Git. Los comandos en Git son muy similares a los que se utilizan en un terminal en Linux, pero para un usuario de Windows, puede ser algo tedioso de aprender. Sin embargo, aprender a usar esta herramienta no es complicado, ya que podemos consultar en linea lo que necesitemos.
\newline
\newline
Una característica resaltante de Github, es que permite que varios usuarios trabajen sobre un mismo proyecto/repositorio, pero, es importante que no trabajen sobre el mismo archivo a la vez, pues los cambios no se podrían efectuar de manera correcta.
	\newline
	\newline	
	\newline
	\begingroup
		\large{
			\textbf{
				Experiencias y Anécdotas: LaTeX
				\newline
				\newline
			}
		}
	\endgroup
	\textbf{Ana:\newline\newline}Latex es una buena opción al momento de la creación de documentos de distinto tipo, una herramienta que nos saca de la rutina de los mismos editores de texto que utilizamos a diario; y sin embargo Latex ofrece las mismas características que dichos editores.
\newline	
\newline	
Latex  a pesar de aparentar dificultad, es una herramienta sencilla de utilizar, solo es cuestión de conocer los comandos que permiten editar el formato de nuestro documento.
Otra ventaja de Latex es que existen plantillas para realizar distintos tipos de documentos, estas plantillas facilitan la edición del formato de un texto que quizás no sepamos cómo debería estar formado.
\newline	
\newline	
Como conclusión, el uso de esta herramienta me pareció entretenida y útil, además, fue un reto ya que como programadores debemos ser capaces de adaptarnos a cualquier herramienta.
\newline
\newline		
Existen muchos tutoriales de cómo utilizar Latex, los cuales fueron de mucha ayuda al momento de generar el primer documento que fue el CV, ya que al ser una herramienta nueva, tuve que introducirme al uso de ésta y cuáles son las mejores cualidades que ofrece; para este artículo tengo más experiencia y me costó menos tiempo hacerlo y se me hizo mucho más entretenido.
\newline
\newline	
\textbf{Denny:\newline\newline} LaTeX es una buena herramienta para elaborar documentos. Muy útil bajo ciertas circunstancias, como la creación de libros (en especial de tipo matemáticos), pero bajo otras es probable que el usuario prefiera una herramienta de la categoría ``WYSIWYG".
\newline
\newline
Aprender a elaborar un documento en LaTeX ha sido relativamente fácil, ya que hay muchos Websites en donde se especifica cómo utilizarlo, y sus comandos son hasta cierto punto intuitivos. Lo que no ha sido fácil, es lograr que el código se compile. Y es que la simple acción de identar el código, o dar saltos de linea,  podria hacer que el documento no se pueda compilar. A esto habría que agregarle, que los errores de compilación son difíciles de interpretar. 
\newline
\newline
LaTex tiene un potencial bastante bueno, en cuanto a la elaboración de documentos que, sin duda, se podría aprovechar muy bien con un poco más de práctica, como la inserción de ecuaciones en la redacción de un libro. Pero en cuanto a la elaboración de presentaciones, necesita un toque más multimedia, para poder satisfacer las necesidades de los usuarios.
\newline
\newline
\textbf{Liliana:\newline\newline}Desde mi punto de vista, LaTeX es una aplicación que se ajusta a las necesidades de alguien que quiere plantillas predeterminadas para sus proyectos, artículos, reportes o presentaciones. Es muy diferente a las herramientas que hemos ulitizado previamente, puesto que aquí lo escribimos en forma de código.
\newline
\newline
Para poder realizar trabajos, es necesario investigar mucho antes de colocar la primera palabra. Claro que esto se vuelve menos tedioso puesto que todo lo podemos encontrar en internet, pero aun así es un poco complicado realizar cosas simples como interlineados, el tamaño y tipo de letra o colocar una imagen.
\newline
\newline
LaTeX tampoco ofrece lo que es correción de ortografía. Y LAS COMILLAS NO SON COMILLAS. En el idioma español es complicado compilar bien el archivos, puesto que hay que colocar dos comillas simples para que simulen ser comillas dobles en el pdf.
\newline
\newline
No todo fue negativo a la hora de realizar un archivo. Algo que sí tiene de positivo, es que lo que uno coloca, se queda siempre en su lugar, no como las otras herramientas que terminan haciendo lo que quieren con las imagenes o texto.
\newline
\newline
Sé que a pesar de que he expuesto cosas no tan buenas para esta aplicación, la volvería a utilizar.
\newline
\newline
\end{document}


